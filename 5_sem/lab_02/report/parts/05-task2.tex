\chapter{Эксперимент 2: Сравнение эффективности ссылочных и векторных структур}

\section{Цель эксперимента}
Оценить влияние зависимости команд по данным на эффективность вычислений.

\section{Описание проблемы}
Обработка зависимых данных происходит в тех случаях, когда результат работы одной команды используется в качестве адреса операнда другой. При программировании на языках высокого уровня такими операндами являются указатели, активно используемые при обработке ссылочных структур данных: списков, деревьев, графов. Обработка данных структур процессорами с длинными конвейерами команд приводит к заметному увеличению времени работы алгоритмов: адрес загружаемого операнда становится известным только после обработки предыдущей команды. В противоположность этому, обработка векторных структур, таких как массивы, позволяет эффективно использовать аппаратные возможности ЭВМ. 

\section{Суть эксперимента}
Для сравнения эффективности векторных и списковых структур в эксперименте применяется профилировка кода двух алгоритмов поиска минимального значения. Первый алгоритм использует для хранения данных список, в то время как во втором применяется массив. Очевидно, что время работы алгоритма поиска минимального значения в списке зависит от его фрагментации, т.е. от среднего расстояния между элементами списка. 

\section{Условия эксперимента}
\begin{enumerate}
    \item Единицы измерения по Ох - Килобайты;
    \item Единицы измерения по Оу - Такты;
    \item Количество элементов в списке: \textbf{1};
    \item Максимальная фрагментация списка: \textbf{256};
    \item Шаг изменения фрагментации: \textbf{4};
\end{enumerate}

\section{Результаты эксперимента}
\begin{figure}[ht!]
    \centering
    \includegraphics[width=170mm]{./img/task_02.png}
    \caption{Эксперимент 2: Сравнение эффективности ссылочных и векторных структур}
\end{figure}

\begin{figure}[ht!]
    \centering
    \includegraphics[width=100mm]{./img/res_02.png}
    \caption{Эксперимент 2: Результаты\label{res_02}}
\end{figure}

Как видно из рисунка \ref{res_02}: массив обрабатывается примерно в 20-30 раз быстрее списка.

\section{Вывод}
Вывод из полученных результатов можно сделать следующий: использовать структуры данных надо с учетом технологического фактора
определенной задачи. Если решаемая задача предполагает возможность использования массива, то надо использовать его, особенно если использование списка не дает существенной разницы (особенно выигрыша во времени).
