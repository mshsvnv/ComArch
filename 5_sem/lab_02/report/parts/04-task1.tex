% \chapter{Задания} 
\chapter{Эксперимент 1: Исследование расслоения динамической памяти}

\section{Цель эксперимента}
Определить способ трансляции физического адреса, используемый при обращении к динамической памяти.  

\section{Описание проблемы}
В связи с конструктивной неоднородностью  оперативной памяти,  обращение к последовательно расположенным данным  требует различного времени. В связи с этим, для создания эффективных программ необходимо учитывать расслоение памяти, характеризуемое способом трансляции физического адреса. 

\section{Суть эксперимента}
Для определения способа трансляции физического адреса при формировании сигналов выборки банка, выборки строки и столбца запоминающего массива применяется процедура замера времени обращения к динамической памяти по последовательным адресам с изменяющимся шагом чтения. Для сравнения времен используется обращение к одинаковому количеству различных ячеек,  отстоящих друг от друга  на  определенный шаг. Результат эксперимента представляется зависимостью  времени (или количества тактов процессора), потраченного на чтение ячеек от шага чтения. 

\section{Условия эксперимента}
\begin{enumerate}
    \item Единицы измерения по Ох - Байты;
    \item Единицы измерения по Оу - Такты;
    \item Максимальное расстояние между читаемыми данными: \textbf{32};
    \item Шаг увеличения расстояния между читаемыми 4х байтовыми ячейками: \textbf{64};
    \item Размер массива: \textbf{8};
\end{enumerate}

\section{Результаты эксперимента}
\begin{figure}[ht!]
    \centering
    \includegraphics[width=170mm]{./img/task_01.png}
    \caption{Эксперимент 1: Исследование расслоения динамической памяти\label{overflow}}
\end{figure}

\section{Вывод}
Оперативная память неоднородна, и для обращения к последовательно расположенными данным может потребоваться различное количество
времени. Поэтому, при создании программ необходимо учитывать расслоение памяти при обработке данных.