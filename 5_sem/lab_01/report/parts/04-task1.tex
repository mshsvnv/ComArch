\chapter{Задания} 
\section{Задание 1}

Содержимое файла \textit{mine.s}, который соответсвует \textbf{варианту 17}:
\begin{lstlisting}
    .section .text
    .globl _start;
    len = 9 #Размер массива
    enroll = 2 #Количество обрабатываемых элементов за одну итерацию
    elem_sz = 4 #Размер одного элемента массива

_start:
    la x1, _x
    addi x20, x1, elem_sz*len #Адрес элемента, следующего за последним
    lw x31, 0(x1)
    addi x1, x1, elem_sz*1
lp:
    lw x2, 0(x1)
    lw x3, 4(x1) #!
    bltu x2, x31, lt1
    add x31, x0, x2
lt1:    bltu x3, x31, lt2
    add x31, x0, x3 
lt2:
    add x1, x1, elem_sz*enroll
    bne x1, x20, lp
lp2: j lp2

    .section .data
_x: .4byte 0x1
    .4byte 0x2
    .4byte 0x3
    .4byte 0x4
    .4byte 0x5
    .4byte 0x6
    .4byte 0x7
    .4byte 0x8
    .4byte 0x9
    
\end{lstlisting}

Псевдокод:
\begin{lstlisting}{language=C}
#define len 9
#define enroll 2
#define elem_sz 4

int _x[] = [1, 2, 3, 4, 5, 6, 7, 8, 9];

void start() {
    int *x1 = _x;
    int *x20 = x1 + elem_sz * len;

    int x31 = _x[0];
    int x1 += 1;

    do {
        x2 = x1[0];
        x3 = x1[1];

        if (x2 >= x31)
            x31 = x2;
        
        if (x3 >= x31)
            x31 = x3;
        
        x1 += enroll;
    } while (x1 != x20);
    while (1) {}
}
\end{lstlisting}

Результат выполнения компиляции:
\begin{lstlisting}{}
Disassembly of section .text:

80000000 <_start>:
80000000:       00000097                auipc   x1,0x0
80000004:       03808093                addi    x1,x1,56 # 80000038 <_x>
80000008:       02408a13                addi    x20,x1,36
8000000c:       0000af83                lw      x31,0(x1)
80000010:       00408093                addi    x1,x1,4

80000014 <lp>:
80000014:       0000a103                lw      x2,0(x1)
80000018:       0040a183                lw      x3,4(x1)
8000001c:       01f16463                bltu    x2,x31,80000024 <lt1>
80000020:       00200fb3                add     x31,x0,x2

80000024 <lt1>:
80000024:       01f1e463                bltu    x3,x31,8000002c <lt2>
80000028:       00300fb3                add     x31,x0,x3

8000002c <lt2>:
8000002c:       00808093                addi    x1,x1,8
80000030:       ff4092e3                bne     x1,x20,80000014 <lp>

80000034 <lp2>:
80000034:       0000006f                jal     x0,80000034 <lp2>

\end{lstlisting}

Содержимое файла \textit{mine.hex:}
\begin{lstlisting}
00000097
03808093
02408a13
0000af83
00408093
0000a103
0040a183
01f16463
00200fb3
01f1e463
00300fb3
00808093
ff4092e3
0000006f
00000001
00000002
00000003
00000004
00000005
00000006
00000007
00000008
00000009
\end{lstlisting}

В регистре x31 в конце выполнения программы должно содержаться \textbf{максимальное значение} в массива. В нашем случае: \textbf{9}.
